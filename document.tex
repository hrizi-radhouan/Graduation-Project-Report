\documentclass[12pt]{article}
% include the pckage of the color%
\usepackage[usenames, dvipsnames]{color}
\usepackage[english]{babel}
\usepackage[utf8x]{inputenc}
\usepackage{amsmath}
\usepackage{graphicx}

%define your own color %
\definecolor{mygray}{gray}{0.9}

\begin{document}
	
	
	\title{Chapter 4 : Implementation}
	\maketitle
	\section{Introduction}
	
	In this chpater we will focus on the technologies. 
	
	\section{Angular JS}
	
	
	% Include logo of AngularJS %
	\begin{figure}[h]
		\centering
		\includegraphics[width=0.25\textwidth]{angularjsLogo.png}
		\caption{Logo AngularJS}
		\label{fig:mesh1}
	\end{figure}
	
	AngularJS is a structural framework for dynamic web apps. It lets you use HTML as your template language and lets you extend HTML's syntax to express your application's components clearly and succinctly. AngularJS's data binding and dependency injection eliminate much of the code you would otherwise have to write.  And it all happens within the browser, making it an ideal partner with any server technology.
	\\
	\\
	AngularJS is what HTML would have been, had it been designed for applications. HTML is a great declarative language for static documents. It does not contain much in the way of creating applications, and as a result building web applications is an exercise \textit{in what do I have to do to trick the browser into doing what I want?}
	\\
	\\
	The impedance mismatch between dynamic applications and static documents is often solved with:
	\begin{itemize}
		
		\item \textbf{a library} - a collection of functions which are useful when writing web apps. Your code is in charge and it calls into the library when it sees fit. E.g., \colorbox{mygray}{jQuery}.
		\item \textbf{frameworks} - a particular implementation of a web application, where your code fills in the details. The framework is in charge and it calls into your code when it needs something app specific. E.g., \colorbox{mygray}{durandal}, \colorbox{mygray}{ember}, etc.
	\end{itemize}
	AngularJS takes another approach. It attempts to minimize the impedance mismatch between document centric HTML and what an application needs by creating new HTML constructs. AngularJS teaches the browser new syntax through a construct we call directives. Examples include:
	\begin{itemize}
		\item Data binding, as in \{\{\}\}
		\item DOM control structures for repeating, showing and hiding DOM fragments.
		\item Support for forms and form validation.
		\item Attaching new behavior to DOM elements, such as DOM event handling.
		\item Grouping of HTML into reusable components.
	\end{itemize}
	
	
	
	\listoffigures
	
\end{document}